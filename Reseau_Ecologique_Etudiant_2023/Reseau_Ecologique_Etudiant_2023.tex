\documentclass[9pt,twocolumn,twoside,]{pnas-new}

% Use the lineno option to display guide line numbers if required.
% Note that the use of elements such as single-column equations
% may affect the guide line number alignment.


\usepackage[T1]{fontenc}
\usepackage[utf8]{inputenc}

% tightlist command for lists without linebreak
\providecommand{\tightlist}{%
  \setlength{\itemsep}{0pt}\setlength{\parskip}{0pt}}




\templatetype{pnasresearcharticle}  % Choose template

\title{Réseau écologique des étudiants de Sherbrooke}

\author[a]{Yuriko Archambault}
\author[a]{Marie-Ève Gagné}
\author[a]{Juliette Larrivée}
\author[a]{Daphnée Longworth}

  \affil[a]{Université de Sherbrooke, Faculté de science, 2500 Bd de l'Univeristé,
Sherbrooke, Québec, J1K 2R1}


% Please give the surname of the lead author for the running footer
\leadauthor{}

% Please add here a significance statement to explain the relevance of your work
\significancestatement{}


\authorcontributions{}



\correspondingauthor{\textsuperscript{} }

% Keywords are not mandatory, but authors are strongly encouraged to provide them. If provided, please include two to five keywords, separated by the pipe symbol, e.g:
 \keywords{  one |  two |  optional |  optional |  optional  } 

\begin{abstract}
Please provide an abstract of no more than 250 words in a single
paragraph.
\end{abstract}

\dates{This manuscript was compiled on \today}
\doi{\url{www.pnas.org/cgi/doi/10.1073/pnas.XXXXXXXXXX}}

\begin{document}

% Optional adjustment to line up main text (after abstract) of first page with line numbers, when using both lineno and twocolumn options.
% You should only change this length when you've finalised the article contents.
\verticaladjustment{-2pt}



\maketitle
\thispagestyle{firststyle}
\ifthenelse{\boolean{shortarticle}}{\ifthenelse{\boolean{singlecolumn}}{\abscontentformatted}{\abscontent}}{}

% If your first paragraph (i.e. with the \dropcap) contains a list environment (quote, quotation, theorem, definition, enumerate, itemize...), the line after the list may have some extra indentation. If this is the case, add \parshape=0 to the end of the list environment.

\acknow{}

\hypertarget{intro}{%
\section*{Intro}\label{intro}}
\addcontentsline{toc}{section}{Intro}

Texte ici

\hypertarget{ruxe9sultats}{%
\section*{Résultats}\label{ruxe9sultats}}
\addcontentsline{toc}{section}{Résultats}

\hypertarget{section-1}{%
\subsection*{Section 1}\label{section-1}}
\addcontentsline{toc}{subsection}{Section 1}

texte ici

\hypertarget{section-2}{%
\subsection*{Section 2}\label{section-2}}
\addcontentsline{toc}{subsection}{Section 2}

texte ici

\hypertarget{section-3}{%
\subsection*{Section 3}\label{section-3}}
\addcontentsline{toc}{subsection}{Section 3}

texte ici

\hypertarget{muxe9thode}{%
\section*{Méthode}\label{muxe9thode}}
\addcontentsline{toc}{section}{Méthode}

\hypertarget{section-1-1}{%
\subsection*{Section 1}\label{section-1-1}}
\addcontentsline{toc}{subsection}{Section 1}

texte ici

\hypertarget{section-2-1}{%
\subsection*{Section 2}\label{section-2-1}}
\addcontentsline{toc}{subsection}{Section 2}

texte ici

\hypertarget{section-3-1}{%
\subsection*{Section 3}\label{section-3-1}}
\addcontentsline{toc}{subsection}{Section 3}

Texte ici

\hypertarget{discussion}{%
\section*{Discussion}\label{discussion}}
\addcontentsline{toc}{section}{Discussion}

\hypertarget{section-1-2}{%
\subsection*{Section 1}\label{section-1-2}}
\addcontentsline{toc}{subsection}{Section 1}

ici

\hypertarget{section-2-2}{%
\subsection*{Section 2}\label{section-2-2}}
\addcontentsline{toc}{subsection}{Section 2}

ici

\hypertarget{section-3-2}{%
\subsection*{Section 3}\label{section-3-2}}
\addcontentsline{toc}{subsection}{Section 3}

ici

\hypertarget{conclusion}{%
\section*{Conclusion}\label{conclusion}}
\addcontentsline{toc}{section}{Conclusion}

\hypertarget{references}{%
\section*{Bibliographie}\label{references}}
\addcontentsline{toc}{section}{Bibliographie}

References should be cited in numerical order as they appear in text;
this will be done automatically via bibtex, e.g.~(1) and (2, 3). All
references, including for the SI, should be included in the main
manuscript file. References appearing in both sections should not be
duplicated. SI references included in tables should be included with the
main reference section.

\hypertarget{figure-et-tableau}{%
\subsection*{figure et tableau}\label{figure-et-tableau}}
\addcontentsline{toc}{subsection}{figure et tableau}

\begin{figure}
\centering
\includegraphics{frog.png}
\caption{Placeholder image of a frog with a long example caption to show
justification setting.{}}
\end{figure}

Figures and Tables should be labelled and referenced in the standard way
using the \texttt{\textbackslash{}label\{\}} and
\texttt{\textbackslash{}ref\{\}} commands.

Figure \[fig:frog\] shows an example of how to insert a column-wide
figure. To insert a figure wider than one column, please use the
\texttt{\textbackslash{}begin\{figure*\}...\textbackslash{}end\{figure*\}}
environment. Figures wider than one column should be sized to 11.4 cm or
17.8 cm wide.

\[\begin{aligned}
(x+y)^3&=(x+y)(x+y)^2\\
       &=(x+y)(x^2+2xy+y^2) \label{eqn:example} \\
       &=x^3+3x^2y+3xy^3+x^3. 
\end{aligned}\]

\showmatmethods
\pnasbreak

\hypertarget{refs}{}
\leavevmode\hypertarget{ref-belkin2002using}{}%
1. Belkin M, Niyogi P (2002) Using manifold stucture for partially
labeled classification. \emph{Advances in Neural Information Processing
Systems}, pp 929--936.

\leavevmode\hypertarget{ref-berard1994embedding}{}%
2. Bérard P, Besson G, Gallot S (1994) Embedding riemannian manifolds by
their heat kernel. \emph{Geometric \& Functional Analysis GAFA}
4(4):373--398.

\leavevmode\hypertarget{ref-coifman2005geometric}{}%
3. Coifman RR, et al. (2005) Geometric diffusions as a tool for harmonic
analysis and structure definition of data: Diffusion maps.
\emph{Proceedings of the National Academy of Sciences of the United
States of America} 102(21):7426--7431.



% Bibliography
% \bibliography{pnas-sample}

\end{document}

