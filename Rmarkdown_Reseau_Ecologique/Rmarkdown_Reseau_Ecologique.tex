\documentclass[9pt,twocolumn,twoside,]{pnas-new}

% Use the lineno option to display guide line numbers if required.
% Note that the use of elements such as single-column equations
% may affect the guide line number alignment.


\usepackage[T1]{fontenc}
\usepackage[utf8]{inputenc}

% tightlist command for lists without linebreak
\providecommand{\tightlist}{%
  \setlength{\itemsep}{0pt}\setlength{\parskip}{0pt}}




\templatetype{pnasresearcharticle}

\title{Réseau écologique des étudiants de Sherbrooke}




% Please give the surname of the lead author for the running footer
\leadauthor{}

% Please add here a significance statement to explain the relevance of your work
\significancestatement{}


\authorcontributions{}



\correspondingauthor{\textsuperscript{} }

% Keywords are not mandatory, but authors are strongly encouraged to provide them. If provided, please include two to five keywords, separated by the pipe symbol, e.g:


\begin{abstract}

\end{abstract}

\dates{This manuscript was compiled on \today}
\doi{\url{www.pnas.org/cgi/doi/10.1073/pnas.XXXXXXXXXX}}

\begin{document}

% Optional adjustment to line up main text (after abstract) of first page with line numbers, when using both lineno and twocolumn options.
% You should only change this length when you've finalised the article contents.
\verticaladjustment{-2pt}



\maketitle
\thispagestyle{firststyle}
\ifthenelse{\boolean{shortarticle}}{\ifthenelse{\boolean{singlecolumn}}{\abscontentformatted}{\abscontent}}{}

% If your first paragraph (i.e. with the \dropcap) contains a list environment (quote, quotation, theorem, definition, enumerate, itemize...), the line after the list may have some extra indentation. If this is the case, add \parshape=0 to the end of the list environment.

\acknow{}

\section*{Auteurs}

Yuriko Archambault \textbackslash{} Marie-Ève Gagné \textbackslash{}
Juliette Larrivée \textbackslash{} Daphnée Longworth \textbackslash{}

\hypertarget{introduction}{%
\section{Introduction}\label{introduction}}

Texte ici

\hypertarget{ruxe9sultats}{%
\section{Résultats}\label{ruxe9sultats}}

\hypertarget{section-1}{%
\subsection{Section 1}\label{section-1}}

Texte ici

\hypertarget{section-2}{%
\subsection{Section 2}\label{section-2}}

Texte ici

\hypertarget{section-3}{%
\subsection{Section 3}\label{section-3}}

Texte ici

\hypertarget{muxe9thode}{%
\section{Méthode}\label{muxe9thode}}

\hypertarget{section-1-1}{%
\subsection{Section 1}\label{section-1-1}}

Texte ici

\hypertarget{section-2-1}{%
\subsection{Section 2}\label{section-2-1}}

Texte ici

\hypertarget{section-3-1}{%
\subsection{Section 3}\label{section-3-1}}

Texte ici

\hypertarget{discussion}{%
\section{Discussion}\label{discussion}}

\hypertarget{section-1-2}{%
\subsection{Section 1}\label{section-1-2}}

Texte ici

\hypertarget{section-2-2}{%
\subsection{Section 2}\label{section-2-2}}

Texte ici

\hypertarget{section-3-2}{%
\subsection{Section 3}\label{section-3-2}}

Texte ici

\hypertarget{conclusion}{%
\section{Conclusion}\label{conclusion}}

\hypertarget{bibliographie}{%
\section{Bibliographie}\label{bibliographie}}

\hypertarget{figure-et-tableau}{%
\section{Figure et tableau}\label{figure-et-tableau}}

\begin{figure}
\centering
\includegraphics{frog.png}
\caption{Nom de la figure.{}}
\end{figure}



% Bibliography
% \bibliography{pnas-sample}

\end{document}
